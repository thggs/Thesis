%!TEX root = ../template.tex
%%%%%%%%%%%%%%%%%%%%%%%%%%%%%%%%%%%%%%%%%%%%%%%%%%%%%%%%%%%%%%%%%%%%
%% abstrac-en.tex
%% NOVA thesis document file
%%
%% Abstract in English([^%]*)
%%%%%%%%%%%%%%%%%%%%%%%%%%%%%%%%%%%%%%%%%%%%%%%%%%%%%%%%%%%%%%%%%%%%

\typeout{NT FILE abstrac-en.tex}%

The concept of Industry 4.0 revolutionized the manufacturing sector. It called for the use of Cyber-Physical Production Systems and the digitalization of many manufacturing processes. Even though the implementation of Industrial Multi-agent Systems as Cyber-Physical Production Systems comes with a lot of advantages, it has not seen much use by the industry.

As a consequence of this, Multi-agent Systems have not grown out of their infancy and haven't evolved like other technologies through their practical use. This may be because there is still some skepticism surrounding the concept of Multi-agent Systems, and whether or not they can perform to the same efficacy when compared to the already existing systems.

In this work, an architecture which helps solve this problem is proposed. More precisely, this platform allows for the flexible integration of agents with their respective hardware by proposing a method that selects one or more generic libraries based on the kind of hardware that is being integrated into the industrial Multi-agent System. By doing this, more devices are able to be integrated in less time and with less work, contributing for the flexibility and scalability that is characteristic of Multi-agent Systems.

This platform would facilitate the adoption of industrial Multi-agent Systems, because not only would it make integrating them easier, it would also mean that anyone could write these libraries, making it easily adaptable to already existing systems, which would reduce costs in the adoption of Multi-agent Systems for industrial applications.

%
%The abstracts' order varies with the school.  If your school has specific regulations concerning the abstracts' order, the \gls{novathesis} (\LaTeX) template will respect them.  Otherwise, the default rule in the \gls{novathesis} template is to have in first place the abstract in \emph{the same language as main text}, and then the abstract in \emph{the other language}. For example, if the dissertation is written in Portuguese, the abstracts' order will be first Portuguese and then English, followed by the main text in Portuguese. If the dissertation is written in English, the abstracts' order will be first English and then Portuguese, followed by the main text in English.
%
%However, this order can be customized by adding one of the following to the file \verb+5_packages.tex+.

%\begin{verbatim}
%    \ntsetup{abstractorder={<LANG_1>,...,<LANG_N>}}
%    \ntsetup{abstractorder={<MAIN_LANG>={<LANG_1>,...,<LANG_N>}}}
%\end{verbatim}
%
%For example, for a main document written in German with abstracts written in German, English and Italian (by this order) use:
%\begin{verbatim}
%    \ntsetup{abstractorder={de={de,en,it}}}
%\end{verbatim}
% Palavras-chave do resumo em Inglês
% \begin{keywords}
% Keyword 1, Keyword 2, Keyword 3, Keyword 4, Keyword 5, Keyword 6, Keyword 7, Keyword 8, Keyword 9
% \end{keywords}
\keywords{
  Industry 4.0 \and
  Cyber-Physical Production System \and
  Multi-agent System \and
  Manufacturing Systems \and
  Hardware Integration
}
