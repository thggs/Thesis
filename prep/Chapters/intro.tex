\typeout{NT FILE intro.tex}

\prependtographicspath{{Chapters/Figures/}}

\chapter{Introduction}
\label{cha:introduction}

\section{Motivation}
\label{sec:motivation}

In 2011 the term Industrie 4.0 was introduced for the first time during a German conference. This term was later translated into Industry 4.0 and it quickly became synonymous with the fourth industrial revolution. With it, came the appearance of \gls{CPPS}. The concept of a \gls{CPPS}, is essentially a \gls{CPS} that is thought of from a production perspective. A \gls{CPS} is a physical system that has a digital representation. Both the physical and digital counterparts exchange data to maintain coherence and work in tandem to achieve a goal. These \gls{CPPS}s brought about the digitalization of the manufacturing sector, and despite there being different models, most of them follow these design principles \cite{birgit01}:

\begin{itemize}
	\item Services offered through an online platform
	\item Decentralized, enabling autonomous decision-making
	\item Virtualized, to allow interoperability
	\item Modular, making them flexible to changes in the system
	\item Real-time capabilities
	\item Ability to optimize processes
	\item Communication is done through secure channels
	\item Cloud service for data storage and management
\end{itemize}

This new way of operating a production chain revolutionized the industry, because it brought about the changes needed to make manufacturing processes more flexible, adaptable and re-configurable, bringing with it a solution for the ever increasing complexity needed to address the rapidly changing costumer demands.
In recent years, multiple companies have made the transition to this model. They have made diversified changes, from the way they analyze and process data to the way they manufacture and distribute products, with very positive results as we can see in \cite{rit01}. \\

In light of this, \gls{CPPS}s became a popular research subject. These \gls{CPPS}s would allow for more efficient, robust and flexible systems, equipped with Big Data analyzing algorithms, Cloud storage to easily access data, service oriented manufacturing and interoperability.\\

This research was mainly done on the topic of \gls{MAS} \cite{sakurada01} \cite{karnouskos01}. These \gls{MAS}s are a coalition of agents, all part of one single system but completely independent of each other. They are intelligent, social and capable of performing tasks on their own, however due to their social capabilities, they can also perform tasks cooperatively, making them powerful tools in goal-oriented networks. Because this system is, by nature, decentralized, these agents can leave and join a coalition of agents as needed, to complete selfish or collective objectives. Evidently, the system is also highly flexible and robust, since agents can be taken out of commission and new agents can be introduced as needed, either because the overall system specifications need to change or simply because an agent has become faulty \cite{paulo02}. \\

\gls{MAS}s have actually been around for decades and as such, a lot of research and standardization already exists, like the \gls{FIPA} specifications. \gls{FIPA} as an organization have ceased operations, however their standard are still put to use in \gls{MAS}s nowadays \cite{FIPA_website}. An industrial \gls{MAS} follows similar requirements as a non-industrial one, although it needs to take into consideration other factors, such as hardware integration, reliability, fault-tolerance and maintenance and management costs.\\

Despite all the advantages an \gls{MAS} has for the manufacturing sector, it has not seen much success outside research fields. This could be due to the fact that there is still skepticism surrounding agent-based systems for industrial production \cite{bottling_plant_part2}. Because it never left the prototyping stages, real-world applications never evolved past their infancy, therefore never gained much momentum in the industry at large \cite{karnouskos02}. Another consequence of this is the difficulty in designing a robust and scalable interface, that allows, for example, the addition and removal of agents without system reconfiguration. 

\subsection{Problem and Solution}

In this work, a new approach to developing an interface for an Industrial \gls{MAS} is proposed. It consists of a framework that allows the creation of industrial agents through the selection of generic libraries for the interface between agent and device. This would simplify the process of creating new agents for the system, making the addition of new agents way more flexible and less time consuming. It would also allow the creation of more generic agents that could be used in to integrate many types of hardware.