% !TEX root=template.tex

\typeout{NT FILE intro.tex}

\prependtographicspath{{Chapters/Figures/}}

\chapter{Introduction}
\label{cha:introduction}

In the Introduction, the motivations for this work are explained. The concept of \acrlongpl{CPPS} is analysed, what they are and how they are useful for the manufacturing industry. The concept of \acrlongpl{MAS} is also explored and why they represent \acrlongpl{CPPS}. Then some issues that have arisen since the conceptualization of Industrial \acrlongpl{MAS} are identified and a solution to help tackle them is proposed.

\section{Motivation}
\label{sec:motivation}

In 2011 the term Industrie 4.0 was introduced for the first time during a German conference. This term was later translated into Industry 4.0 and it quickly became synonymous with the fourth industrial revolution \cite{birgit01}. With it, came the appearance of \acrfullpl{CPPS}. A \acrshort{CPPS} is essentially, as the name implies, a \acrfull{CPS} capable of operating in an industrial setting. A \acrshort{CPS} is a physical system that has a digital representation. Both the physical and digital counterparts exchange data to maintain coherence and work in tandem to achieve a goal. These \acrshortpl{CPPS} brought about the digitalization of the manufacturing sector, and despite there being different models, most of them follow these design principles \cite{birgit01}:

\begin{itemize}
	\item Services offered through an online platform;
	\item Decentralized, enabling autonomous decision-making;
	\item Virtualized, to allow interoperability;
	\item Modular, making them flexible to changes in the system;
	\item Real-time capabilities;
	\item Ability to optimize processes;
	\item Communication is done through secure channels;
	\item Cloud service for data storage and management.
\end{itemize}

This new way of operating a production chain revolutionized the industry, because it brought about the changes needed to make manufacturing processes more flexible, adaptable and re-configurable, bringing with it a solution for the ever increasing complexity needed to address the rapidly changing costumer demands.
In recent years, multiple companies have made the transition to this model. They have made diversified changes, from the way they analyse and process data to the way they manufacture and distribute products, with very positive results identified in \cite{rit01}.\\

In light of this, \acrlongpl{CPPS} became a popular research subject. These \acrshortpl{CPPS} would allow for more efficient, robust and flexible systems, equipped with Big Data analyzing algorithms, Cloud Storage to easily access data, service oriented manufacturing and interoperability.\\

This research was mainly done on the topic of \acrfullpl{MAS} \cite{sakurada01} \cite{karnouskos01}. These \acrshortpl{MAS} are a coalition of agents, all part of one single system but completely independent of each other. They are intelligent, social and capable of performing tasks on their own, however due to their social capabilities, they can also perform tasks cooperatively, making them powerful tools in goal-oriented networks. Because this system is, by nature, decentralized, these agents can leave and join a coalition of agents as needed, to complete selfish or collective objectives. Evidently, the system is also highly flexible and robust, since agents can be taken out of commission and new agents can be introduced as needed, either because the overall system specifications need to change or simply because an agent has become faulty \cite{paulo02}. \\

\acrlongpl{MAS} have actually been around for decades and as such, a lot of research and standardization already exists, like the \gls{FIPA} specifications. \gls{FIPA} as an organization have ceased operations, however their standard are still put to use in \acrshortpl{MAS} nowadays \cite{FIPA_website}. An industrial \acrshort{MAS} follows similar requirements as a non-industrial one, although it needs to take into consideration other factors, such as hardware integration, reliability, fault-tolerance and maintenance and management costs.\\

\section{Problem and Solution}
\label{sec:problem_and_solution}

Despite all the advantages an \acrshort{MAS} has for the manufacturing sector, it has not seen much success outside research fields. This could be due to the fact that there is still scepticism surrounding agent-based systems for industrial production \cite{bottling_plant_part2}. Because \acrshortpl{MAS} never left the prototyping stages, real-world applications never evolved past their infancy. Therefore, they never gained much momentum in the industry at large \cite{karnouskos02}. Another consequence of this is the difficulty in designing a robust and scalable interface, that allows, for example, the addition and removal of agents without system reconfiguration.\\

In this work, a new approach to developing interfaces for an Industrial \acrshort{MAS} is proposed. It consists of a framework that allows the creation of industrial agents through the selection of generic libraries for the interface between agent and device. These libraries are picked based on the type of communication protocols the hardware implements and because they are generic, they can be re-used and replaced with minimal reconfigurations. The aim of this solution is to simplify the process of creating new agents for the system, enabling them to interface with any kind of hardware. This would minimize development time and costs when creating an Industrial \acrshort{MAS}, making them a more viable option when implementing a \acrshort{CPPS}. 