% !TEX root=template.tex

\typeout{NT FILE conclusion.tex}

\prependtographicspath{{Chapters/Figures/}}

\glsresetall

\chapter{Conclusion}
\label{cha:conclusion}

The concept of Industry 4.0 revolutionized the manufacturing sector through the digitalization of manufacturing processes. Industrial \acrlong{CPS} or \acrlongpl{CPPS} enable the transition to the Industry 4.0 standard. These systems are service oriented, can process Big Data and have cloud integration for storage.\\

They are also able to interface with the real world, by using sensors, and to act on it, by using actuators. In essence these systems are composed of two counterparts, a physical one and a digital one. This makes them very robust and efficient, since they rely on the digital version to extract information on how to act on their environment.\\

Industrial \acrlong{MAS} were suggested as a model for the implementation of a \acrlong{CPPS} due to the advantages they could bring to the industry, such as the decentralization of the system, allowing for the autonomous behaviour of each individual agent to accomplish a common goal. This would make the system very robust and flexible because errors wouldn't propagate throughout the different layers of the system and agents can join and leave the system as needed. They, however, have not seen practical uses outside research prototypes. This may come from the scepticism that they won't perform to the same capabilities as existing systems.\\

In this work, a method to facilitate the integration of Industrial \acrlong{MAS} was presented. The \acrlong{ME} and \acrlongpl{LL} combo allow for a faster and easier interface development, when compared to traditional hardware interfaces. These \acrlongpl{LL} could be developed to support all kinds of protocols and hardware, facilitating the integration of a \acrlong{CPPS}.\\

The prototype presented in this work shows a lot of promise, with minimal drawbacks to performance, it is a great alternative to the methodologies employed nowadays for the integration of hardware on a \acrlong{MAS}. The next step for the development of the \acrlong{ME} would be to add more functionalities, like the ability to change \acrlongpl{LL} without having to terminate an agent, either through an user interface or automatically according to the systems needs. The creation of a web application capable of providing and updating \acrlongpl{LL} remotely and automatically would also be a great feature to add to this framework, enabling an easier exchange of \acrlongpl{LL} between developers and system managers.\\

An interface similar to a \acrlong{LL} could also be implemented between the agent and the \acrlong{ME} to separate the agent from the hardware even further and allow for the on the fly switching of agents without terminating hardware connections and even to permit other agents working with other agent frameworks to interface with the \acrlong{ME}, increasing system modularity. Still, the framework developed in this work is a step in the right direction for the wide adoption of \acrlong{MAS} with industrial applications.