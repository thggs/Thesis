\typeout{NT FILE conclusion.tex}

\prependtographicspath{{Chapters/Figures/}}

\glsresetall

\chapter{Conclusion}
\label{cha:conclusion}

The concept of Industry 4.0 revolutionized the manufacturing sector through the digitalization of manufacturing processes. Industrial \gls{CPS} or \gls{CPPS}s enable the transition to the Industry 4.0 standard. These systems are service oriented, can process Big Data and have cloud integration for storage. They are also able to interface with the real world, by using sensors, and to act on it, by using actuators. In essence these systems are composed of two counterparts, a physical one and a digital one. This makes them very robust and efficient, since they rely on the digital version to extract information on how to act on their environment.\\

Industrial \gls{MAS} were suggested as a model for the implementation of a \gls{CPPS} due to the advantages they could bring to the industry, such as the decentralization of the system, allowing for the autonomous behavior of each individual agent to accomplish a common goal. This makes the system very robust and flexible because errors don't propagate throughout the different layers of the system and agents can join and leave the system as needed. They, however, have not seen practical uses outside research prototypes. This may come from the skepticism that they won't perform to the same capabilities as existing systems.\\

There have been a lot of proposed architectures and methods of interfacing throughout the years to try and make \gls{MAS}s more accessible. This work proposes another such method, enabling more seamless integration of new industrial agents into an already existing system. By selecting a collection of generic libraries from a database during agent integration, it would be possible to create more agents more easily in less time, facilitating the adoption of \gls{MAS} by the industry. This in turn would help the integration of \gls{MAS}s in industrial settings, increasing their use by the manufacturing sector and consequently allowing for the development of even better technologies due to their practical use.